\hypertarget{course-description}{%
\section{Course Description}\label{course-description}}

From the CofC catalog:

\begin{quote}
An introductory course for majors and non-majors that emphasizes the
analysis of current domestic and international issues. Issues covered
will vary from semester to semester.
\end{quote}

\noindent For this course we will be examining the issue of
\textbf{climate change}. Specifically, we will use the issue of climate
change to examine the process of addressing complex problems in the
United States (and globally to a lesser extent), explore how science
impacts policymaking, and examine why the United States has been unable,
at the federal level, to development policy approaches to address
climate change.

\hypertarget{course-goals-and-learning-objectives}{%
\subsection{Course Goals and Learning
Objectives}\label{course-goals-and-learning-objectives}}

\hypertarget{political-science-learning-outcomes}{%
\subsubsection{Political Science Learning
Outcomes}\label{political-science-learning-outcomes}}

After taking this course:

\begin{itemize}
\item
  Students will become familiar with a number of contemporary political
  issues and better understand their make‐up and importance
\item
  Students will understand how different political issues are assessed
  by different philosophical and ideological traditions
\item
  Students will be able to effectively write and develop arguments
\item
  Students will be able to better comprehend other's views and
  formulate, defend their own positions
\end{itemize}

\hypertarget{general-social-science-education-learning-outcomes}{%
\subsubsection{General Social Science Education Learning
Outcomes}\label{general-social-science-education-learning-outcomes}}

Upon completion of this course students should be able to apply social
science concepts, models or theories to explain human behavior, social
interactions or social institutions. This will be assessed in the final
exam.

\hypertarget{delivery-format}{%
\subsection{Delivery Format}\label{delivery-format}}

This is an asynchronous online course, and so it is largely self-paced.
Students must have access to a \textbf{computer} with \textbf{high-speed
internet access} throughout the course. In addition, students must have
access to \textbf{OAKS} and \emph{should check OAKS frequently (AT LEAST
every other day) to be sure not to fall behind}. Finally, students must
have access to their \textbf{CofC email}. \textbf{Computer
failure/unavailability does not constitute an excuse for not completing
assignments by the due date}.

\vspace{0.1in}

\noindent It is essential that students stay on top of the course
assignments. I will post due dates, but it is your responsibility to
make sure you don't get behind, especially in a class this short. Do not
make the mistake of thinking this is an easy class because we're meeting
online, or an easy class because it's meeting over the summer. The
material is quite difficult, and will take a lot of effort on your part
to master. \emph{A Summer I class that meets face-to-face normally
entails three hours of classroom time per weekday, plus reading and
homework each night. The workload for this class will be the same,
except our classroom will be OAKS}.

\hypertarget{technical-issues}{%
\subsubsection{Technical Issues}\label{technical-issues}}

If you have technical problems, please contact the Student Computing
Support Desk at 843.953.8000 or email
\url{studentcomputingsuport@cofc.edu}.

\hypertarget{contacting-the-professor}{%
\subsection{Contacting the Professor}\label{contacting-the-professor}}

If you have questions about course related material, and/or course
procedures please \emph{post your question to the Course Questions
discussion board on} \href{https://lms.cofc.edu/d2l/login}{OAKS}, so
that other students can benefit from your questions and the answer. I
will respond to discussion board questions within 24 hours, \emph{if not
sooner}. If you are having problems with \emph{course material}, please
feel free to email me at \url{nowlinmc@cofc.edu}.

\hypertarget{email-policy}{%
\subsubsection{Email Policy}\label{email-policy}}

Email is the best way to contact me and I am happy to answer questions
and/or address concerns over email. Please note the following:

\begin{enumerate}
\def\labelenumi{\arabic{enumi}.}

\item
  Please allow 24 hours for a response from me before sending a second
  email.
\item
  Assignments must be turned in through the corresponding Assignment
  folder on OAKS and will not be accepted by email under any
  circumstances, including ``issues with OAKS.''
\item
  If you are having a technical issue with OAKS, I will not be able to
  help you, so please contact the
  \href{https://it.cofc.edu/help/studentcomputing.php}{Student Computing
  Support Desk}.
\end{enumerate}

\hypertarget{required-materials}{%
\section{Required Materials}\label{required-materials}}

\begin{itemize}
\item
  Access to \href{https://lms.cofc.edu/d2l/login}{OAKS}. We will make
  extensive use of OAKS in this course and several of its tools
  including Discussion Boards, Quizzes, and Assignments. Tutorials for
  each of these tools can be found
  \href{https://blogs.cofc.edu/sits/tutorials/oaks_tutorials/}{here}.
\item
  \emph{Readings}:

  \begin{itemize}
  
  \item
    \emph{Book}: Dessler, Andrew Emory, and Edward Parson. 2020.
    \emph{The Science and Politics of Global Climate Change: A Guide to
    the Debate}. Cambridge: Cambridge University Press.
  \item
    Additional readings will be available on
    \href{https://lms.cofc.edu/d2l/login}{OAKS}.
  \end{itemize}
\end{itemize}

\hypertarget{navigating-this-course}{%
\section{Navigating This Course}\label{navigating-this-course}}

Course material will be organized into 4 content modules that you will
be able to access on OAKS beginning each Tuesday starting on Tuesday,
June 8th. Each module will consist of:

\begin{itemize}
\item
  Course lectures and/or lecture notes
\item
  Video
\item
  Discussion boards
\item
  Module quiz
\end{itemize}

Each module will be made available at 7:00am on each Tuesday, and
assignments within each module are due by \emph{11:59pm EST on each
Monday}.

\hypertarget{assignments-and-due-dates}{%
\subsection{Assignments and Due Dates}\label{assignments-and-due-dates}}

Your grade in this course will be determined by your performance on 4
module quizzes, 8 discussion board posts, 4 current events, a mid-term
exam, an En-Roads assignment, and a final exam. Detailed instructions
for each assignment will be available on OAKS.

\vspace{0.15in}

\begin{tabular}{l | r | r}
\hline
Assignment & Possible Points & \% of Grade \\
\hline
Module quizzes & 100 (total) & 20\%  \\
Discussion boards & 50 (total) & 10\%  \\
Current Events & 100 (total) & 20\% \\
Mid-term exam & 100 & 20\% \\
En-ROADS assignment & 50 & 10\% \\
Final exam  & 100 & 20\% \\
\hline
\textbf{Total} & 500 & 100\% \\
\hline
\end{tabular}

\hypertarget{module-quizzes}{%
\subsubsection{Module Quizzes}\label{module-quizzes}}

Each of the 4 modules will have a 5 question quiz over the
\textbf{readings} and \textbf{lectures} for that module. You can use
course materials for the quiz, but you must be take each quiz by
yourself. \emph{Once you begin the quiz you will 5 minutes to complete
it}. Each quiz is worth up to \emph{25 points}. \emph{Module quizzes are
due by 11:59pm EST on each Monday}.

\hypertarget{discussion-boards}{%
\subsubsection{Discussion Boards}\label{discussion-boards}}

Within each module there will be a discussion board that will involve a
discussion question from the video(s). For each discussion board you
must a) provide an \emph{answer} to the discussion question and b)
\emph{comment} on one other students answer. Each time you post, it
should be about a paragraph. Note that you will not be able to comment
on another student's post until you provide your answer to the question.
Each post is worth up \emph{6.25 points}. \emph{Discussions will end at
11:59pm EST on each Monday}.

\vspace{0.1in}

\noindent \emph{You should not make any statement to or about anyone in
an email or discussion board that you would not make in person. Be
respectful of your classmates. In this course we are addressing an issue
that is politically controversial. All discussions will be respectful to
differing views. However, this course will be about learning to approach
policy issues as scholars and analysts, not as partisans for one
particular point of view.}

\hypertarget{current-events}{%
\subsubsection{Current Events}\label{current-events}}

Each week you will and summarize a current news story related to climate
change. You \textbf{must} find a story from the
\href{https://www.nytimes.com/}{New York Times},
\href{https://www.washingtonpost.com/}{Washington Post}, or the
\href{https://www.postandcourier.com/}{Post and Courier}. Once you find
a story, you will post a one paragraph summary of the story, a one
paragraph discussion relating the story to a concept within the module,
and a link to the story on the Current Events discussion board.
\emph{Each current event post is worth 25 points}.

\hypertarget{mid-term-exam}{%
\subsubsection{Mid-term exam}\label{mid-term-exam}}

The mid-term exam will be available on OAKS starting Monday June 21st at
7:00am and \emph{must be completed by 11:59pm EST on June 21st}. The
exam will be posted and taken in OAKS. Exam questions will consist of
four short essay questions covering material from module's 1 and 2
including lectures, readings, videos, and/or course discussion from the
discussion boards within each module. \emph{You are allowed to use
course material for the exam, however, you must take the exam on your
own}.

\hypertarget{en-roads-assignment}{%
\subsubsection{En-ROADS assignment}\label{en-roads-assignment}}

\href{https://www.climateinteractive.org/tools/en-roads/}{En-ROADS} is a
web-based tool that allows you to experiment with different ways to
reduce greenhouse-gas emissions. For this assignment you will need to
develop and report on a strategy that successfully keeps estimated
warming to the Paris Agreement goal of 2 degrees Celsius. Further
instructions will be provided on \href{https://lms.cofc.edu}{OAKS}.
\emph{The En-ROADS assignment is worth up to 50 points and is due on
Tuesday July 6th by 11:59 PM}.

\hypertarget{final-exam}{%
\subsubsection{Final exam}\label{final-exam}}

The final exam will be available on OAKS starting Wednesday July 7th at
7:00am and \emph{must be completed by 11:59pm EST on July 7th}. The exam
will be posted and taken in OAKS. Exam questions will consist of four
short essay questions covering material from module's 3 and 4 including
lectures, readings, videos, and/or course discussion from the discussion
boards within each module. \emph{You are allowed to use course material
for the exam, however, you must take the exam on your own}. The final is
\textbf{not} comprehensive and will only cover material from module's 3
and 4.

\hypertarget{grades}{%
\subsection{Grades}\label{grades}}

There are \textbf{500} possible points for this course. Grades will be
allocated based on your earned points and calculated as a percentage of
\textbf{500}. A: 94 to 100\%; A-: 90 to 93\%; B+: 87 to 89\%; B: 83 to
86\%; B-: 80 to 82\%; C+: 77 to 79\%; C: 73 to 76\%; C-: 70 to 72\%; D+:
67 to 69\%; D: 63 to 67\%; D-: 60 to 62\%; F: 59\% and below.

\hypertarget{course-content}{%
\section{Course Content}\label{course-content}}

The following modules will be on OAKS and made available on the dates
indicated. Course content and schedule is subject to change. Changes
will be announced through email.

\hypertarget{module-1-the-science-of-climate-change}{%
\subsection{Module 1: The Science of Climate
Change}\label{module-1-the-science-of-climate-change}}

\hypertarget{available-700-am-est-on-june-8th}{%
\paragraph{Available 7:00 AM EST on June
8th}\label{available-700-am-est-on-june-8th}}

The first module provides a brief overview of climate science and how
scientists know that climate change is happening as a result of human
activity.

\vspace{0.1in}

\noindent See the Module 1 guide on \href{https://lms.cofc.edu}{OAKS}
for a complete list of readings and assignments

\hypertarget{module-2-values-science-and-the-political-controversy-of-climate-change}{%
\subsection{Module 2: Values, Science, and the Political Controversy of
Climate
Change}\label{module-2-values-science-and-the-political-controversy-of-climate-change}}

\hypertarget{available-700am-est-on-june-15th}{%
\paragraph{Available 7:00am EST on June
15th}\label{available-700am-est-on-june-15th}}

The second module considers the role of science in policymaking as well
as the role of differing values and beliefs across stakeholders in the
development of policies to address climate change. Specifically, this
module will examine the question, \emph{given the strong scientific
consensus about climate change, why don't we agree and why don't we
act}?

\vspace{0.1in}

\noindent See the Module 2 guide on \href{https://lms.cofc.edu}{OAKS}
for a complete list of readings and assignments

\begin{itemize}

\item
  \textbf{Mid-Term Exam available on June 21 from 7:00am to 11:59pm}
\end{itemize}

\hypertarget{module-3-climate-policy}{%
\subsection{Module 3: Climate Policy}\label{module-3-climate-policy}}

\hypertarget{available-700am-est-on-june-22nd}{%
\paragraph{Available 7:00am EST on June
22nd}\label{available-700am-est-on-june-22nd}}

The third module examines policy approaches to address climate change at
the global, national (United States), and sub-national level.

\vspace{0.1in}

\noindent See the Module 3 guide on \href{https://lms.cofc.edu}{OAKS}
for a complete list of readings and assignments

\hypertarget{module-4-climate-politics-and-the-road-ahead}{%
\subsection{Module 4: Climate Politics and the Road
Ahead}\label{module-4-climate-politics-and-the-road-ahead}}

\hypertarget{available-700am-est-on-june-29th}{%
\paragraph{Available 7:00am EST on June
29th}\label{available-700am-est-on-june-29th}}

The fourth module explores the current state of climate change politics
and the future \emph{might} hold for the development of climate
policies.

\vspace{0.1in}

\noindent See the Module 4 guide on \href{https://lms.cofc.edu}{OAKS}
for a complete list of readings and assignments

\begin{itemize}
\item
  \textbf{En-ROADS assignment due on July 6th by 11:59 PM}
\item
  \textbf{Final Exam available on July 7th from 7:00am to 11:59pm}
\end{itemize}
